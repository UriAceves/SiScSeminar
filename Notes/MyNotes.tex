%%%%%%%%%%%%%%%%%%%%%%%%%%%%%%%%%%%%%%%%%
% Simple Sectioned Essay Template
% LaTeX Template
%
% This template has been downloaded from:
% http://www.latextemplates.com
%
% Note:
% The \lipsum[#] commands throughout this template generate dummy text
% to fill the template out. These commands should all be removed when
% writing essay content.
%
%%%%%%%%%%%%%%%%%%%%%%%%%%%%%%%%%%%%%%%%%

%----------------------------------------------------------------------------------------
%	PACKAGES AND OTHER DOCUMENT CONFIGURATIONS
%----------------------------------------------------------------------------------------

\documentclass[12pt,twoside,a4paper]{article} % Default font size is 12pt, it can be changed here

\usepackage{geometry} % Required to change the page size to A4
\geometry{legalpaper, margin=2.5cm}


\usepackage{graphicx} % Required for including pictures

\usepackage{float} % Allows putting an [H] in \begin{figure} to specify the exact location of the figure
\usepackage{wrapfig} % Allows in-line images such as the example fish picture
\usepackage[english]{babel}
\usepackage[utf8]{inputenc}
\usepackage{hyperref}
\usepackage{lipsum} % Used for inserting dummy 'Lorem ipsum' text into the template
\usepackage{amssymb}
\usepackage{fancyhdr}
\pagestyle{fancy}
\fancyhead[RO,LE]{\thepage}
\fancyhead[LO]{SIMULATION SCIENCES SEMINAR NOTES}
\fancyhead[RE]{\leftmark}

\linespread{1.0} % Line spacing

%\setlength\parindent{0pt} % Uncomment to remove all indentation from paragraphs

\graphicspath{{Pictures/}} % Specifies the directory where pictures are stored

\renewcommand{\thesection}{\Roman{section}}
\renewcommand{\thesubsection}{\thesection.\Roman{subsection}}

\begin{document}

%----------------------------------------------------------------------------------------
%	TITLE PAGE
%----------------------------------------------------------------------------------------

\begin{titlepage}

\newcommand{\HRule}{\rule{\linewidth}{0.5mm}} % Defines a new command for the horizontal lines, change thickness here

\center % Center everything on the page

\textsc{\LARGE RWTH Aachen University}\\[1.5cm] % Name of your university/college
\textsc{\Large Simulation Sciences seminar}\\[0.5cm] % Major heading such as course name
\textsc{\large Sensitivity analysis on chaotic dynamical systems by NILSS}\\[0.5cm] % Minor heading such as course title

\HRule \\[0.4cm]
{ \huge \bfseries Personal notes for the seminar}\\[0.4cm] % Title of your document
\HRule \\[1.5cm]

\begin{minipage}{0.4\textwidth}
\begin{flushleft} \large
\emph{Author:}\\
Uriel \textsc{Aceves} % Your name
\end{flushleft}
\end{minipage}
~
\begin{minipage}{0.4\textwidth}
\begin{flushright} \large
\emph{Supervisor:} \\
Dr. Johannes \textsc{Lotz} % Supervisor's Name
\end{flushright}
\end{minipage}\\[4cm]

{\large \today}\\[3cm] % Date, change the \today to a set date if you want to be precise

\includegraphics{logo.png}\\[1cm] % Include a department/university logo - this will require the graphicx package

\vfill % Fill the rest of the page with whitespace

\end{titlepage}

%----------------------------------------------------------------------------------------
%	TABLE OF CONTENTS
%----------------------------------------------------------------------------------------

\tableofcontents % Include a table of contents

\newpage % Begins the essay on a new page instead of on the same page as the table of contents

%----------------------------------------------------------------------------------------
%	INTRODUCTION
%----------------------------------------------------------------------------------------


\section*{Disclaimer}
This notes are with the sole intention of better understanding the article that we will use for the simulation sciences seminar \cite{Wang}, and clarify some aspects in more depth. So they will be redundant with the cited paper, and in most cases the phrases will be either paraphrasing or literal copies of the original ones. For this reason the constant citations to \cite{Wang} will be omited, but we urge the reader to keep in mind that we are deliberately ommiting them.
\section{Abstract}
On this paper the Non-Intrusive Least Squares Shadowing (NILSS) is developed, the so called method computes the sensitivity\footnote{Sensitivity is the quantity that allows us to know how small changes on the initial conditions will affect the system for long time periods.} of quantities that are being averaged over long periods of time on chaotic dynamical systems. In NILSS tangent solutions\footnote{God knows what is a tangent solution} are represented as a linear combination of one inhomogeneous tangent solution and several homogeneous tangent solutions, in other words we take the decomposition of such tangent solution in homogeneous and inhomogeneous parts. The next step is to solve a least squares problem\footnote{Minimizing the sum of the squares of some quantities} using this decomposition. And hence the resulting solution can be used to compute the sensitivity\footnote{It remains to see how}. Advantages of this method is that it is easy to implement on existing solvers. furthermore, it has a low computational cost for systems with many degrees of freedom but few unstable modes\footnote{How do we know this and why? Additionaly, what is an unstable mode?}. On the paper NILSS is applied to two systems, Lorenz 63 and CFD simulation of flow over a backward-facing step. As a result, in both cases the sensitivities computed by NILSS reflect the common behaviour in the long-time averaged quantities of dynamical systems\footnote{I don't really know what the common behaviour is}.

\section{Introduction}
Many important phenomena in engineering and sciences are described by chaotic systems. In these systems, we are often interested in properties obtained from long-time averaged quantities. The main problem is, the systems are chaotic, and as a consequence, small changes in the initial data can lead to very different behaviours as time increases. The aim of the paper, is then, to provide tools to analyse the sensitivities of such systems, i.e. how a change on the initial data or the parameters affects the properties we are interested on.

In order to define the problem we pose the differential equation governing the phenomenon and the initial conditions,

\begin{equation}
\frac{du}{dt} = f(u,s), \quad u(t=0) = u_0(\phi),
\end{equation}
where $f(u,s): \mathbb{R}^m \times \mathbb{R} \rightarrow \mathbb{R}^m$\footnote{I need to think about that m on the dimension.} is a smooth function, $u$ is the state of the system, and $s$ is the parameter. The initial condition $u_0$ is a smooth function of $\phi$\footnote{What is $\phi$}. Finally, a solution $u(t)$ will be called the primal solution.

In the paper the objetives are long-time averaged quantities. To define them let $J(u,s): \mathbb{R}^m \times \mathbb{R} \rightarrow \mathbb{R}$ be a continuous function representing the instantaneous objective function. Then we can obtain the objective by averaging over an infinitely long trajectory

\begin{equation}
  \langle J \rangle_\infty := \lim_{t\to\infty} \langle J \rangle_T, \textrm{ where } \langle J \rangle_T := \frac{1}{T}\int_0^T J(u,s)dt,
\end{equation}
$\langle J \rangle_T$ depends on $u$, $s$ and $\phi$, while $\langle J \rangle_\infty$ is determined only by $s$ and $u_0$\footnote{Why not from $\phi$}. If we assume ergodicity, i.e. $\langle J \rangle_T = \langle J \rangle_\infty$, this will mean that $u_0$, and as a consequence $\phi$ does not affect $\langle J \rangle_\infty$\footnote{I don't really understand why}, and as a result $\langle J \rangle_\infty$ only depends on $s$.

The mathematical definition of sensitivity is $d\langle J \rangle_\infty / ds$. Such sensitivity can be useful to solve inverse problems, estimate simulation errors, quantify uncertanties and other things\footnote{Sources are cited on the paper, I might want to go and read superficially some of them just to know}.

When the system is chaotic the computation of sensitivity is challenging, since

\begin{equation}
  \frac{d}{ds} \langle J \rangle_\infty \neq \lim_{t\to \infty} \frac{\partial}{\partial s} \langle J \rangle_T (s,\phi,T).
\end{equation}
This means that, if we fix the initial conditions $u_0(\phi)$, the process of $T\to\infty$ does not commute with differentiation by $s$\footnote{Basically on the left hand side first we let $T\to\infty$ and then differentiate, on the right hand side first we differentiate and then let $T\to\infty$}. And as a result, the transient method does not converge to the correct sensitivity.

Many methods have been developed to compute sensitivities. The usual ones are finite differences and trasient method. There is also an ensemble method, which computes the sensitivity by averaging from the trasient method over an ensemble of trajectories.

One method that has proven to work correctly is Least Squares Shadowing, developed by Wang, Hu and Blonigan. Such method computes a bounded shift of a trajectory under an infinitesimal parameter change\footnote{Read or ask about this to at least know what is going on}. LSS has been succesfully applied to systems such as Lorenz 63 and a modified Kuramoto-Sivashinsky equation. Moreover, Wang has proven that, under assumptions of ergodicity and hyperbolicity, LSS converges to the correct sensitivity at a rate of $T^{-0.5}$, where $T$ is the trajectory time lenght\footnote{Basically as the inverse of the square root.}.

However using LSS is expensive, since it involves solving a linear system of $d*m\tau$, where $d$ is the dimension of the system and $m\tau$ the number of timesteps. In addition to this, it also requires the jacobian $\partial_uf(u,s)$ at each time step. And we consider this as a disadvantage since not all software provides the jacobian, and modify existing codes to include this calculation might not be trivial.

To tackle this problem the paper develops the Non-intrusive Least Squares Shadowing (NILSS). The computational and memory costs of NILSS are proportional to the number of positive Lyapunov Exponents (LE)\footnote{Read a little bit about LE}. NILSS also requires less modifications than LSS to be implemented on existing solvers since it does not requires the jacobian matrix.\footnote{Read about the jacobian}.

The rest of the paper follows this path: First, examine the long-time and trasient effects due to perturbations in the system parameters. It is also examined how trasient effects are generated by changes on the initial conditions. Next they describe the NILSS method as a procedure that gets the long-time effect from the trasient effect, where perturbations are represented by tangent solutions. Then a step-by-step description of the algorithm is presented. And finally the algorithm is applied to two systems.

\section{Questions}
\begin{description}
  \item [1] What is a tangent solution?
  \item [2] What is a least squares problem?
  \item [3] What is an unstable mode?
  \item [4] What are the Lorenz 63 and the CFD simulation of flow over backward-facing step systems?
  \item [5] Why is NILSS good for systems with may degrees of freedom and few unstable modes?
\end{description}
\newpage
\section{To do}

\begin{enumerate}
  \item Find what tangent solutions are.
  \item Explain how the final solution is used to compute sensitivities.
\end{enumerate}



\bibliography{MyNotes.bib}
\bibliographystyle{ieeetr}
%
% Example citation \cite{Figueredo:2009dg}.
%
% %------------------------------------------------
%
% \subsection{Subsection 1} % Sub-section
%
% \lipsum[1] % Dummy text
%
% %------------------------------------------------
%
% \subsection{Subsection 2} % Sub-section
%
% \lipsum[2] % Dummy text
%
% %------------------------------------------------
%
% \subsubsection{Subsubsection 1} % Sub-sub-section
%
% \lipsum[3] % Dummy text
%
% \begin{figure}[H] % Example image
% \center{\includegraphics[width=0.5\linewidth]{placeholder}}
% \caption{Example image.}
% \label{fig:speciation}
% \end{figure}
%
% %------------------------------------------------
%
% \subsubsection{Subsubsection 2} % Sub-sub-section
%
% \lipsum[4] % Dummy text
%
% %----------------------------------------------------------------------------------------
% %	MAJOR SECTION 1
% %----------------------------------------------------------------------------------------
%
% \section{Content Section} % Major section
%
% \lipsum[5] % Dummy text
%
% %------------------------------------------------
%
% \subsection{Subsection 1} % Sub-section
%
% \subsubsection{Subsubsection 1} % Sub-sub-section
%
% \lipsum[6] % Dummy text
%
% %------------------------------------------------
% \newpage
% \subsubsection{Subsubsection 2} % Sub-sub-section
%
% \lipsum[6] % Dummy text
% \begin{wrapfigure}{l}{0.4\textwidth} % Inline image example
%   \begin{center}
%     \includegraphics[width=0.38\textwidth]{fish}
%   \end{center}
%   \caption{Fish}
% \end{wrapfigure}
% \lipsum[7-8] % Dummy text
%
% %------------------------------------------------
%
% \subsubsection{Subsubsection 3} % Sub-sub-section
%
% \begin{description} % Numbered list example
%
% \item[First] \hfill \\
% \lipsum[9] % Dummy text
%
% \item[Second] \hfill \\
% \lipsum[10] % Dummy text
%
% \item[Third] \hfill \\
% \lipsum[11] % Dummy text
%
% \end{description}
%
% %----------------------------------------------------------------------------------------
% %	MAJOR SECTION X - TEMPLATE - UNCOMMENT AND FILL IN
% %----------------------------------------------------------------------------------------
%
% %\section{Content Section}
%
% %\subsection{Subsection 1} % Sub-section
%
% % Content
%
% %------------------------------------------------
%
% %\subsection{Subsection 2} % Sub-section
%
% % Content
%
% %----------------------------------------------------------------------------------------
% %	CONCLUSION
% %----------------------------------------------------------------------------------------
%
% \section{Conclusion} % Major section
%
% \lipsum[12-13]

%----------------------------------------------------------------------------------------
%	BIBLIOGRAPHY
%----------------------------------------------------------------------------------------

%\begin{thebibliography}{99} % Bibliography - this is intentionally simple in this template
%
% \bibitem[Figueredo and Wolf, 2009]{Figueredo:2009dg}
% Figueredo, A.~J. and Wolf, P. S.~A. (2009).
% \newblock Assortative pairing and life history strategy - a cross-cultural
%   study.
% \newblock {\em Human Nature}, 20:317--330.
%
% \end{thebibliography}

%----------------------------------------------------------------------------------------

\end{document}
