\documentclass{beamer}
\usetheme{JUB}

\usepackage[utf8]{inputenc}
\usepackage[T1]{fontenc}
\usepackage[scaled]{helvet}
\usepackage{media9}

%% Use any fonts you like.
% \usepackage{libertine}

\title{Embracing the Chaos}
\subtitle{Sensitivity Analysis on Chaotic Dynamical Systems by NILSS}
\author{Uriel A. Aceves R.}
\date{June 20, 2018\\ \vspace{0.3cm}
Supervisor: Dr. Johannes Lotz (STCE, RWTH Aachen)}
\institute{\url{uriel.aceves@rwth-aachen.de}}

\begin{document}

\begin{frame}[plain,t]
\titlepage
\end{frame}

\begin{frame}% [plain,t]
	\frametitle{Outline}
  \begin{footnotesize}
\tableofcontents
\end{footnotesize}
\end{frame}

%=============================================================================================
\section{Introduction}
  \subsection{Chaos as a way of living}

\begin{frame}
  \frametitle{Should the world behave nicely?}
  \textit{
  “Chaos was the law of nature; Order was the dream of man.”
  ― Henry Adams}

\end{frame}

\section{Predictability in chaos?}
\subsection{So what about butterflies?}
  \Transition{images/lorenz}{https://pbs.twimg.com/media/C75sWjvW0AA8Mfc.jpg}

\begin{frame}
  \frametitle{The Lorenz equations}
  The Lorenz 63 system is a 3 variable system with three parameters $\sigma$, $\rho$, and $\beta$.
  \begin{eqnarray}
    \frac{dx}{dt} = \sigma (y-x)\\
    \frac{dy}{dt} = x (\rho -z) -y\\
    \frac{dz}{dt} = xy - \beta z
  \end{eqnarray}
\end{frame}

\subsection{I have seen this before}
  \begin{frame}
    \frametitle{Getting Closer}
    \begin{center}
    \includegraphics[width=0.38\textwidth]{images/002_very_far.png}
    \includegraphics[width=0.38\textwidth]{images/002_close.png}\\

    \includegraphics[width=0.38\textwidth]{images/003_closer.png}
    \includegraphics[width=0.38\textwidth]{images/004_very_close.png}
  \end{center}
  \end{frame}

  \subsection{Oh no... Nevermind}

  \Transition{images/butter}{http://www.mrlovenstein.com/comic/50}

  \subsection{Should we give up?}
  \begin{frame}
    \frametitle{Let's Focus}
    \begin{center}
    \includegraphics[width=0.7\textwidth]{images/001_lage.png}\\
    \tiny{Three highlighted zones}
  \end{center}
  \end{frame}

\begin{frame}
  \frametitle{There is hope after all}
  \begin{center}
  \includegraphics[width=\textwidth]{images/average.png}\\
  \tiny{Time spent on average around this zones}
\end{center}
\end{frame}

\Transition{images/optimism}{https://www.onlinecollegecourses.com/2012/06/21/why-optimism-matters-for-student-success-now-and-after-graduation-2/}

\section{Non-Intrusive Least Squares Shadowing}
\subsection{We have to be careful}
\begin{frame}
  \frametitle{Dynamical systems and sensitivities}
  The governing equation of a dynamical system is
  \begin{equation}
    \frac{du}{dt} = f(u,s), \quad{u(t=0)=u_0},
  \end{equation}
  We want to analyze the changes of a long-time averaged quantity represented by $J(u,s)$.
  \begin{equation}
    \langle J \rangle_\infty := \lim_{t\rightarrow \infty} \frac{1}{T}\int_0^T J(u,s)dt.
  \end{equation}
\end{frame}

\begin{frame}
  \frametitle{It doesn't look that hard}
  We want to calculate $\frac{d}{ds}\langle J \rangle_\infty $ the problem is...\pause
  \begin{equation}
    \frac{d}{ds} \langle J \rangle_\infty \neq \lim_{T\rightarrow \infty} \frac{\partial}{\partial s} \langle J \rangle_T (s,\phi,T).
  \end{equation}
  \pause
  The usual methods diverge most of the time, sometimes they exceed by $10^{100}$ the expected value.\footnote{This competes hand in hand with string theory for the prize for worst predictions of all time.}
\end{frame}

\Transition{images/spider}{http://knowyourmeme.com/}

\subsection{Exploit similarities}

\begin{frame}
  \frametitle{Vary initial conditions vs time evolution}
  \begin{center}

    \includegraphics[width=.98\textwidth]{images/transient}\\


\tiny{Ni, A., Wang, Q., (2017), \textit{Sensitivity analysis on chaotic dynamical systems by Non-Intrusive Least Squares Shadowing (NILSS)} , Journal of Computational Physics, \textbf{347}, 56-77.}
\end{center}
\end{frame}

\Transition{images/smil}{http://www.insurancechat.co.za/2017-09/could-sending-a-smiley-face-get-me-into-legal-hot-water/}

\begin{frame}
  \frametitle{Subtract instabilities}

  \begin{itemize}
    \item We are looking to build trajectories with parameters $\rho$ and $\rho + \delta \rho$ such that they don't diverge from each other.\pause
    \item Their difference contains only the long-time effect.
    \item   Therefore we can reveal the long time effect with shorter trajectories.
  \end{itemize}
\end{frame}

\begin{frame}
  \frametitle{Decomposing our equation}
  To detect variations due changes on the parameters we differentiate the governing equation and decouple into homogeneous and inhomogeneous part
  \begin{equation}
    \frac{dv^*}{dt} - \partial_u fv^* = \partial_s f,
  \end{equation}
  \begin{equation}
    \frac{dw}{dt} - \partial_u fw = 0.
  \end{equation}
\end{frame}

\begin{frame}
  \frametitle{Idea}
  \begin{center}

    \includegraphics[width=0.6\textwidth]{images/intuition}\\

    Now we only need to solve a minimization problem.
    \begin{equation}
      v^\perp = v^{*\perp} + W^\perp a \approx v^{\infty \perp},
    \end{equation}
    \begin{equation}
      \min_a \frac{1}{2} \int_0^T  (v^{*\perp} + W^\perp a)^T (v^{*\perp} + W^\perp a)
    \end{equation}

    \tiny{Ni, A., Wang, Q., (2017), \textit{Sensitivity analysis on chaotic dynamical systems by Non-Intrusive Least Squares Shadowing (NILSS)} , Journal of Computational Physics, \textbf{347}, 56-77.}

\end{center}
\end{frame}

\subsection{Algorithm}
\begin{frame}
  \frametitle{Flowchart}
  \begin{center}

    \includegraphics[width=\textwidth]{images/diag}\\


\end{center}
\end{frame}

\Transition{images/smMex}{https://www.pinterest.de/pin/128211920620408724}

\subsection{Examples}

\begin{frame}
  \frametitle{Lorenz attractor}
  \begin{center}

    \includegraphics[width=0.85\textwidth]{images/finite}\\

  \includegraphics[width=0.40\textwidth]{images/nils}
  %\includegraphics[width=0.45\textwidth]{images/VdPol}

\tiny{Patrick J. Blonigan, Qiqi Wang, Eric J. Nielsen, and Boris Diskin.  \"Least-Squares Shadowing Sensitivity Analysis of Chaotic Flow Around a Two-Dimensional Airfoil\", AIAA Journal, Vol. 56, No. 2 (2018), pp. 658-672.}
\end{center}
\end{frame}

% \begin{frame}
%   \frametitle{Van der Pol attractor}
%   \begin{center}
%
%     %\includegraphics[width=0.85\textwidth]{images/finite}\\
%
%   %\includegraphics[width=0.40\textwidth]{images/nils}
%   \includegraphics[width=0.70\textwidth]{images/VdPol}
%
%
% \end{center}
% \end{frame}

\begin{frame}
  \frametitle{One more example}
  \begin{center}

    %\includegraphics[width=0.85\textwidth]{images/finite}\\

  %\includegraphics[width=0.40\textwidth]{images/nils}
  3D Flow past a cylinder
  \includegraphics[width=0.70\textwidth]{images/cyl}\\

  Cylinder can rotate\\

\end{center}

\tiny{Angxiu Ni, Qiqi Wang, Pablo Fernandez, Chaitanya Talnikar. Sensitivity analysis on chaotic dynamical systems by Finite Difference Non-Intrusive Least Squares Shadowing (FD-NILSS).	arXiv:1711.06633 [physics.comp-ph]}
\end{frame}



  \begin{frame}
    \frametitle{One more example}
    \begin{center}


    \includegraphics[width=0.70\textwidth]{images/photo}\\

    Snapshot of the flow\\

  \end{center}

  \tiny{Angxiu Ni, Qiqi Wang, Pablo Fernandez, Chaitanya Talnikar. Sensitivity analysis on chaotic dynamical systems by Finite Difference Non-Intrusive Least Squares Shadowing (FD-NILSS).	arXiv:1711.06633 [physics.comp-ph]}
  \end{frame}

  \begin{frame}

  \frametitle{One more example}
  \begin{center}

    %\includegraphics[width=0.85\textwidth]{images/finite}\\

  %\includegraphics[width=0.40\textwidth]{images/nils}
  \includegraphics[width=0.85\textwidth]{images/bound}\\


\end{center}

\tiny{Angxiu Ni, Qiqi Wang, Pablo Fernandez, Chaitanya Talnikar. Sensitivity analysis on chaotic dynamical systems by Finite Difference Non-Intrusive Least Squares Shadowing (FD-NILSS).	arXiv:1711.06633 [physics.comp-ph]}
\end{frame}

\section{Why should I care?}

\begin{frame}
  \frametitle{Advantages (self proclaimed)}
  \begin{itemize}
    \item Easy to implement if you already have a solver.
    \item Low cost in comparison to other methods.\pause
    \item Therefore faster.
    \item Uses less memory.
  \end{itemize}
\end{frame}

\begin{frame}
  \frametitle{Where are sensitivities used?}
  Sensitivities help us to\\
  \begin{itemize}
    \item Control processes and systems.
    \item Solve inverse problems (e.g. CAT scan images).
    \item Estimate simulation errors.
    \item Quantify uncertainties.
  \end{itemize}
\end{frame}

\Transition{images/meteor}{https://news.nationalgeographic.com/}

\section{References}
\begin{frame}
  \frametitle{More to know}
  \begin{scriptsize}
  \begin{enumerate}
    \item Ni A., Wang Q., (2017), \textit{Sensitivity analysis on chaotic dynamical systems by Non-Intrusive Least Squares Shadowing (NILSS)} , Journal of Computational Physics, \textbf{347}, 56-77.
    \item {Ni} A., {Wang} Q., {Fernandez} P., and {Talnikar} C., \textit{Sensitivity analysis on chaotic dynamical systems by Finite Difference Non-Intrusive Least Squares Shadowing (FD-NILSS)}, \url{arXiv:1711.06633}
    \item Safiran N., Lotz J., Naumann U., (2016), \textit{Algorithmic Differentiation of Numerical Methods:Tangent and Adjoint Solvers for Parameterized Systems of Nonlinear Equations}, Procedia Computer Science, \textbf{80}, 2231-2235.
    \item Strogatz Steven H., (2015). \textit{Nonlinear dynamics and chaos: with applications to physics, biology, chemistry, and engineering.} Boulder, CO: Westview Press.
    \item Gleick J. (1988). \textit{Chaos: Making a new science}. New York, N.Y., U.S.A: Penguin.
    \item Alvarez A., Ghys É., and Leys J., \textit{Chaos a Mathematical Adventure}, \url{http://www.chaos-math.org/en}
  \end{enumerate}
\end{scriptsize}
\end{frame}

\Transition{images/QA}{http://www.sednacomics.com/}


% \section{Introduction}
% \begin{frame}
% \frametitle{A Frame}
% \framesubtitle{Bullet points}
% \begin{itemize}
% \item First thing
% 	\begin{itemize}
% 	\item small point
% 	\item fine print
% 	\end{itemize}
% \item Second thing
% 	\begin{enumerate}
% 	\item point 1
% 	\end{enumerate}
% \item Third thing
% 	\begin{description}
% 	\item[Research] the scientific pursuit for knowledge
% 	\end{description}
% \end{itemize}
% \end{frame}
%
% \subsection{Text}
% \begin{frame}
% \frametitle{Another Frame}
% Lorem ipsum dolor sit amet, consectetur adipisicing elit, sed do eiusmod tempor incididunt ut labore et dolore magna aliqua. Ut enim ad minim veniam, quis nostrud exercitation ullamco laboris nisi ut aliquip ex ea commodo consequat.
% \end{frame}
%
% \subsection{Blocks}
% \begin{frame}
% \frametitle{Blocks}
% \begin{definition}[Greetings]
% Hello World
% \end{definition}
%
% \begin{theorem}[Fermat's Last Theorem]
% $a^n + b^n = c^n, n \leq 2$
% \end{theorem}
%
% \begin{alertblock}{Uh-oh.}
% By the pricking of my thumbs.
% \end{alertblock}
%
% \begin{exampleblock}{Uh-oh.}
% Something evil this way comes.
% \end{exampleblock}
%
% \end{frame}

\ThankYouFrame

\end{document}
